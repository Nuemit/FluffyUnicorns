\documentclass{article}
\usepackage[utf8]{inputenc}
\usepackage{setspace}
\usepackage{array}
\usepackage{graphicx} % Required for inserting images
\usepackage{ulem}
\usepackage[style=apa]{biblatex}
\addbibresource{quellen.bib}

\title{Klausurvorbereitung-GMED-I}
\author{
    Tim Nätebus \\
    Technische Hochschule Brandenburg \\
    Fachbereich: Informatik und Medien \\
    Studiengang: Medizininformatik \\
    \and
    Jonase Giesecke \\
    Technische Hochschule Brandenburg \\
    Fachbereich: Informatik und Medien \\
    Studiengang: Medizininformatik    
}
\date{November 2024}
\begin{document}
\maketitle

\newpage
%
%\section{Begriffserklärung - Regionen}
%\subsection{Regio prästernalis}
%\begin{itemize}
%    \item \textbf{Definition:} als \textbf{\underline{Regio prästernalis}}, 
%    wird in der Anatomie, die Körper\-region bezeichnet, \underline{die vor dem Sternum} liegt.
%    \item \textbf{Anatomie:}   
%\end{itemize}
\newpage
\section{Osmolarität \/ Tonizität}
\subsection{Kochsalzlösung (NaCl)}
\begin{itemize}
    \item \textbf{Hypotone Kochsalzlösung:} \\
    Wenn weniger als 9 Gramm Kochsalz pro Liter Lösung enthalten sind, spricht man von einer Hypotonen NaCl Lösung.
    \\ \underline{$<$ 9gr Pro Liter z.B. 0,45\%ig}
    \item \textbf{Isotone Kochsalzlösung:} \newline 9 Gramm Kochsalz pro Liter Lösung (0,9\%ig)
    \item \textbf{Hypertone Kochsalzlösung:} \newline Wenn mehr als 9 Gramm Kochsalz pro Liter in einer Lösung enthalten sind, spricht man von einer Hypertonen NaCl Lösung.
    \\
    \underline{$>$ 9gr Pro Liter z.B. 10\%ig}
\end{itemize}

\subsection{Hypoton}
\textbf{Wann Bezeichnet man eine Lösung als hypoton?}
Man bezeichnet enine Lösung als "hypoton", wenn sie einen geringeren osmotischen Druck als ein Vergleichsmedium beseitzt. \newline 
\textbf{\textit{Wenn}} eine Lösung eine kleinere Anzahl gelöster Teilchen pro Volumeneinheit als das Vergleichsmedium hat, spricht man von hypoosmolar. \newline
Als Bezugswert wird in der Medizin normalerweise die Osmolarität des Blutplasmas (ca. 290 Osm/L) verwendet.

\subsection{Isoton}
Bedeutet, dass zwei Lösungen den gleichen osmotischen Druck haben.

\subsection{Hyperton}
Man Bezeichnet eine Lösung als Hyperton, wenn Sie einen höheren osmotischen Druck als ein Vergleichsmedium besitzt.

\newpage
\section{Vokabeln}
\subsection{Begriffe-Tabellen}
\begin{table}[ht]
    \caption{Lagebegriffe Einfach}
    \label{tab:Lagebegriffe}
    \resizebox{\textwidth}{!}{%
    \begin{tabular}{|l|l|}
    \hline 
        \textbf{Begriff} & \textbf{Beschreibung} \\ \hline 
        Dexter, Dextra, Dextrum & Rechts\\ \hline 
        Sinister & Links \\ \hline 
        Proximal & Zum Rumpf hin \\ \hline 
        Distal & Von der Körpermitte entfernt \\ \hline 
        Medial & Zur Körpermitte hin Orientiert  \\ \hline 
        Lateral & Seitlich \\ \hline 
        Superior & Oben gelegen \\ \hline 
        Inferior & Unterhalb gelegen \\ \hline 
        Kranial & Zum Kopf hin \\ \hline
        Kaudal & Zum Steißbein hin \\ \hline
        Posterior / Dorsal & Zum Rücken hin \\ \hline
        Anterior / Ventral & Zum Bauch hin \\ \hline
    \end{tabular}%
    }
\end{table}
\begin{table}[hp]
    \caption{Begriffe Knochen}
    \label{tab:Knochenbegriffe}
    \resizebox{\textwidth}{!}{%
    \renewcommand{\arraystretch}{1.2}  % More vertical space
    \setlength{\tabcolsep}{1em}        % More horizontal space
    \begin{tabular}{|l|l|}
    \hline
        \textbf{Begriff} & \textbf{Beschreibung} \\ \hline
        Septum & Trennwand \\ \hline
        Foramina & Loch \\ \hline
        Foramen Magnum & Öffnung im Bereich der hinteren Schädelgrube \\ \hline
        Medulla Oblongata & Verlängertes Mark \\ \hline
        Cavum Nasi & Nasenhöhle \\ \hline
        Ramus & Der Ast \\ \hline
        Corpus  & Der Körper \\ \hline
        Mandibula & Unterkieferknochen \\ \hline
        Os Nasale & Nasenbein \\ \hline
        Atlas (C1) & Erster Halswirbel (C1) \\ \hline
        Axis (C2) &  Zweiter Halswirbel (C2) \\ \hline
        Scapula & Schulterblatt \\ \hline
        Das Akromion & höchster Punkt des Schulterblatts.\\ \hline
        Ossa Carpalia & Handwurzelknochen \\ \hline
        Ossa Metacarpalia & Mittelhandknochen \\ \hline
        Phalanges & Fingerknochen \\ \hline
        Olecranon & Ellenbogenhöcker \\ \hline
        Syndesmose & Unechtes Gelenk \\ \hline
        Partella & Sesambein \\ \hline
        Patella & kniescheibe \\ \hline
        Malleolen & Knöchel \\ \hline
        Malleolus Medialis & Innenknöchel \\ \hline
        Malleolus Lateralis & Außenknöchel \\ \hline
        Membrana Interossea &  Fibröse Bindegewebsschicht zwischen zwei Knochen \\ \hline
        Ossa & Die Knochen \\ \hline
        Ossa Tarsalia & Fußwurzelknochen \\ \hline
        Ossa Metatarsalia & Mittelfußknochen \\ \hline
        Caput Ossis Metatarsal's & Der Kopf des Mittelfußknochens \\ \hline
        Os & Der Knochen \\ \hline
    \end{tabular}%
    }
\end{table}
\newpage
\section{Zellen}
\begin{itemize}
    \item Die Zelle ist der Grunbaustein allen Lebens
    \item Im Zytoplasma der Zelle befinden sich alle verschiedene Zellorganellen
    \item Die Zelle besitzt eine funktionelle Kompartimentierung, d.h. die einzelnen Zellorganellen werden durch Membranstrukturen getrennt.
    \item Das Zytoskelett besteht aus Mikrotubuli, Aktinfilamenten und \\ Intermediärfilamenten.
    \item Jede Zellfraktion besitzt ihr eigenes Leitenzym
\end{itemize}
\subsubsection{Zellbestandteile}
\begin{itemize}
    \item Zytoplasma
    \item Zellkern
    \item Zytoskellet
\end{itemize}
\subsection{Zellverbindungen}
\begin{itemize}
    \item \textbf{Definition:} \\
    Als Zellkontakte bezeichnet man dauerhafte oder temporäre Verbindungen zwischen Zellen bzw. zwischen Zellen und der extrazellulären Matrix.
    \item \textbf{Funktion:} \\
    Zellkontakte ermöglichen
    \begin{itemize}
        \item die Organisation gleichartiger Zellen zu Geweben und
        \item die Adhäsion zwischen Gewebezellen und gewebefremden Zellen (z.B. Leukozytenadhäsion, Adhäsion von Tumorzellen)
    \end{itemize}
    zum Zwecke
    \begin{itemize}
        \item der funktionellen Stabilität und/oder der
        \item zellulären Kommunikation.
    \end{itemize}
    \item \textbf{Formen:} 
    \begin{itemize}
        \item Tight Junctions
        \item Adhäsionskontakte
        \item Gap Junctions
    \end{itemize}
\end{itemize}
\newpage
\subsection{Unterschiede - Desmosom, \\ Hemidesmosom \& Gap junctions}
\subsubsection{Desmosomen}
\begin{itemize}
    \item \textbf{Funktion:} \\
    Desmosomen sind Haftverbindungen zwischen benachbarten Zellen, die mechanische Stabilität verleihen, insbesondere in Geweben, die mechanischer Belastung ausgesetzt sind, wie z. B. Haut und Herzmuskel.
    \item \textbf{Aufbau:} \\
    Sie bestehen aus Proteinen, wie z. B. Cadherinen (Desmoglein und Desmocollin), die an der Zelloberfläche von benachbarten Zellen miteinander verbinden. Im Inneren der Zellen sind sie über Plaque-Proteine an intermediäre Filamente (z. B. Keratin) gebunden, die die Stabilität erhöhen.
    \item \textbf{Ort:} \\
    Typisch in Epithel- und Muskelzellen, wo mechanische Belastung abgefangen werden muss.
\end{itemize}
\subsubsection{Hemidesmosomen}
\begin{itemize}
    \item \textbf{Funktion:}\\
        Hemidesmosomen sind Zell-Matrix-Verbindungen, die die Zellen mit der extrazellulären Matrix (meist Basallamina) verankern und so die Zelle mit dem darunterliegenden Gewebe verbinden.
    \item \textbf{Aufbau:}\\
        Hemidesmosomen enthalten Integrine anstelle von Cadherinen, die die Zellmembran mit der Basallamina verbinden. Sie sind ebenfalls an intermediäre Filamente der Zelle (z. B. Keratin) gebunden.
    \item \textbf{Ort:}\\
        Sie kommen häufig in Epithelzellen vor, die auf einer Basallamina sitzen, z. B. in der Haut.
        
\end{itemize}
\newpage
\subsubsection{Gap Junctions}
\begin{itemize}
    \item \textbf{Funktion:}\\
        Gap Junctions sind Kommunikationsverbindungen zwischen benachbarten Zellen, die den direkten Austausch kleiner Moleküle und Ionen ermöglichen. Dadurch werden die Zellen metabolisch und elektrisch gekoppelt.
    \item \textbf{Aufbau:}\\
        Sie bestehen aus Connexonen, die aus Connexin-Proteinen gebildet werden und Kanäle zwischen den Zellen schaffen. Diese Kanäle können sich öffnen und schließen, um den Austausch zu regulieren.
    \item \textbf{Ort:}\\
        Gap Junctions kommen häufig in Geweben vor, die eine koordinierte Aktivität benötigen, wie z. B. im Herzen, glatten Muskelgewebe und einigen Nervenzellen.
\end{itemize}
\subsubsection{Zusammenfassung}
\begin{table}[h!]
    \centering
    \begin{tabular}{|>{\raggedright\arraybackslash}p{2.2cm}|
        >{\raggedright\arraybackslash}p{3cm}|
        >{\raggedright\arraybackslash}p{2.6cm}|
        >{\raggedright\arraybackslash}p{2.5cm}|}
        \hline
        \textbf{Typ} & \textbf{Funktion} & \textbf{Hauptb\-estandteile} & \textbf{Beispiel\-gewebe} \\
        \hline
        \footnotesize
        Desmosom & Haftverbindung zwischen Zellen, mechanische Stabilität & Cadherine, intermediäre Filamente (z. B. Keratin) & Epithelien, Herzmuskel \\
        \hline
        \footnotesize
        Hemidesmosom & Verankerung der Zelle an der Basallamina & Integrine, intermediäre Filamente & Epithelien (Haut) \\
        \hline
        \footnotesize
        Gap Junction & Zell-Zell-Kommunikation, Austausch kleiner Moleküle & Connexine (Connexone) & Herz, glatte Muskeln, Nervenzellen \\
        \hline
    \end{tabular}
    \caption{Zusammenfassung der Unterschiede zwischen Desmosom, Hemidesmosom und Gap Junction}
\end{table}

\newpage
\subsection{Transkription \& Translation}
\textbf{Transkription: } findet im Zellkern statt und ist der Prozess, bei dem DNA in mRNA umgeschrieben wird. Das Produkt ist mRNA. \newline
\\
\textbf{Translation: } findet im Zytoplasma an den Ribosomen statt und ist der Prozess, bei dem die mRNA in eine Aminosäuresequenz (Protein) übersetzt wird.
\subsection{Mitose}
Die \textbf{Mitose} ist der \underline{Prozess} der \underline{Zellteilung}, bei dem eine \textbf{Mutterzelle} sich in \textbf{zwei identische Tochterzellen} teilt. \newline
Dieser Prozess ist \underline{wichtig für} das \underline{Wachstum} und die \underline{Reparatur} von Geweben.

\subsection{Meiose}
Die \textbf{Meiose} ist ein spezieller \textbf{Teilungsprozess}, der zur Bildung von Geschlechtszellen (\underline{Eizellen und Spermien}) führt. \newline Dabei wird die \underline{Chromosomenanzahl halbiert}, was genetische Vielfalt ermöglicht.

\subsection{Diffusion}
Diffusion ist der passive Transport von Molekülen von einem Bereich hoher Konzentration zu einem Bereich niedriger Konzentration. \newline
\textbf{beeinflusst von: }
\begin{itemize}
    \item Temperatur
    \item Molekülgröße
    \item Konzentrationsgradienten
\end{itemize}

\subsection{Osmose}
Osmose ist die \textbf{Diffusion} von Wasser durch eine \textbf{semipermeable} Membran. \newline
Sie wird \textbf{beeinflusst} von der \underline{Konzentration gelöster Stoffe} auf beiden Seiten der Membran.

\newpage
\subsection{Unterschiede – Aktiver / Passiver Transport}
\begin{itemize}
    \item \textbf{Passiver Transport:} benötigt keine Energie und erfolgt entlang des Konzentrationsgradienten (z.B. Diffusion und Osmose).
    \item \textbf{Aktiver Transport:} benötigt Energie (meist in Form von ATP), um Moleküle gegen den Konzentrationsgradienten zu transportieren.
\end{itemize}

\subsection{Exozytose}
Exozytose ist der Prozess, bei dem Zellen Moleküle in Vesikeln zur Zellmembran transportieren und nach außen abgeben, z.B. bei der Freisetzung von Neurotransmittern.

\subsection{Endozytose} 
Endozytose ist der Prozess, bei dem Zellen Moleküle oder Partikel aus ihrer Umgebung aufnehmen, indem sie die Zellmembran einziehen und Vesikel bilden.

\subsection{Phagozytose}
Phagozytose ist eine Art der Endozytose, bei der die Zelle größere Partikel (z.B. Bakterien) aufnimmt, indem sie sie umschließt und in Vesikel aufnimmt.

\newpage
\subsection{Drüsen}
Eine Drüse kann aus: einer einzelnen Zelle oder einer Zellgruppe bestehen, die auf eine Oberfläche, in Kanäle/Gänge oder ins Blut sezenieren.
\newline
\newline Man unterscheidet in \textbf{exokrine} und \textbf{endokrine} Drüsen.
\newline
\newline
\textbf{exokrine Drüsen:} geben sekretorische Produkte durch Gänge an eine innere oder äußere Körperoberfläche ab.
\begin{itemize}
    \item \textbf{Vorkommen:} 
    \newline Haut: Schweißdrüsen, Talgdrüsen, Ohrenschmalzdrüsen. 
    \newline Mundhöhle: Kleine- und Ohrspeicheldrüse, Unterkieferspeicheldrüse, Unterzungenspeicheldrüse 
    \newline Verdauungsorgan: Bauchspeicheldrüse, Magendrüsen
    \item \textbf{funktionelle Einteilung:} Man unterscheidet aufgrund der Art und Weise, wie das Sekret freigesetzt wird.
    \begin{itemize}
        \item \textbf{merokrine Drüsen:} Sekret wird durch Vesikel nach außen abgegeben ohne dabei Zellbestandteile  zu verlieren, Bsp.: Speicheldrüse, Schweißdrüse
        \item \textbf{apokrine Drüse:} verliert Teil der Zellmembran beim abschnüren des Sekrets von der Zelle. Zelle bleibt funktionsfähig, Bsp.: Duftdrüse
        \item \textbf{holokrine Drüse:} gesamte Zelle wird für Sekretiontsbildung benutzt, um Sekret freizusetzen stirbt die Zelle und löst sich komplett auf, Bsp.: Talgdrüse
    \end{itemize}
\end{itemize}

\normalsize
\newpage
\section{Knochen}
\subsection{Knochenzellen}
\textbf{Im Knochengewebe} findet man drei verschiedene Formen von Zellen:
\begin{itemize}
    \item \textbf{Osteoblasten}\\
    Sie entstehen aus Vorläuferzellen und produzieren die organische Grundsubstanz des Knochens, das Osteoid sowie die alkalische Phosphatase, welche die Mineralisation des Knochens steuert.
    \item \textbf{Osteozyten}\\
    Reife Knochenzellen, die aus Osteoblasten entstehen. Sie kommunizieren über Zellfortsätze miteinander und dienen der Erhaltung der \textbf{Knochenmatrix} und \textbf{Calciumhomöostase}.  
    \item \textbf{Osteoklasten}\\
    Vielkernige Riesenzellen, die sich aus monozytären Stammzelllinien entwickeln. \underline{Sie sind für den Abbau des Knochens verantwortlich.} Sie sind in den Resorptionszonen des Knochens zu finden.
\end{itemize}
\subsection{Biochemie}
Knochen bestehen etwa zu:
\begin{itemize}
    \item 60-70\% aus anorganischen Mineralien
    \item 10-15\% aus Wasser
    \item 20-25\% aus organischer Substanz
\end{itemize}
\subsection{Röhrenknochen}
\textbf{Röhrenknochen} sind Knochen, welche eine \underline{\textit{einheitliche Markhöhle}} haben und dem Namen entsprechend eine längliche Form zeigen. \\
Zu den Röhrenknochen Zählt man unter anderem:
\begin{tabular}{@{}l l@{}}\\
    \hline
    \textbf{den Femur}   & (o. Oberschenkelknochen) \\
    \textbf{die Tibia}   & (o. Schienbein) \\
    \textbf{die Fibula}  & (o. Wadenbein) \\
    \textbf{den Humerus} & (o. Oberarmknochen) \\
    \textbf{den Radius}  & (o. Speiche) \\
    \textbf{die Ulna}    & (o. Elle) \\
    \hline \\
\end{tabular}
\\
Die Epiphysen bilden die beiden Enden des Röhrenknochens und tragen die knorpeligen Gelenkflächen. In diesem Knochenbereich ist die Compacta eher dünn ausgebildet. \\ \\ 
Nach Abschluss des Wachstums (mit etwa dem 20.Lebensjahr) beginnt die Epiphysenfuge zu verknöchern und bleibt folgend als Epiphysenlinie erhalten.
\subsubsection{Frakturen}
Eine \textbf{Fraktur} des Röhrenknochens ist eine Folge übermäßiger mechanischer Belastung des Knochens.
Ursache ist meist eine plötzliche heftige Gewalteinwirkung, welcher der Knochen nicht standhalten kann. 
Die Fraktur kann dabei je nach Ereignis einfach oder mehrfach sowie offen oder geschlossen sein.
\newpage
\setlength{\bibhang}{2em} % Hängender Einzug
\setlength{\bibitemsep}{1em} % Abstand zwischen Einträgen
\nocite{*}
\printbibliography[title={Literaturverzeichnis}]
\end{document}