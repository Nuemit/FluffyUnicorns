\documentclass{article}
\usepackage{amsmath}
\usepackage{amssymb}

\begin{document}

\section*{Disjunktive und Konjunktive Normalform}
Hier ist eine einfache Erklärung der disjunktiven und konjunktiven Normalform, die auch ein Grundschüler verstehen kann, zusammen mit drei Beispielen und einer Schritt-für-Schritt-Anleitung zur Umstellung von einer Form zur anderen.
\subsection*{Disjunktive Normalform (DNF)}
Die disjunktive Normalform ist eine Art, logische Ausdrücke zu schreiben. Sie besteht aus einer Reihe von "ODER"-Verknüpfungen (Disjunktionen) von "UND"-Verknüpfungen (Konjunktionen).\\
\subsection*{Konjunktive Normalform (KNF)}
Die konjunktive Normalform ist eine andere Art, logische Ausdrücke zu schreiben. Sie besteht aus einer Reihe von "UND"-Verknüpfungen (Konjunktionen) von "ODER"-Verknüpfungen (Disjunktionen).\\
\subsection*{Beispiele}
\subsubsection*{Beispiel 1}
\textbf{Ausgangsformel:}
\[
( (A \land B) \lor C )
\]
\textbf{Disjunktive Normalform (DNF):}
Die Formel ist bereits in DNF, da sie eine ODER-Verknüpfung von UND-Verknüpfungen ist.
\textbf{Konjunktive Normalform (KNF):}
Um die Formel in KNF umzuwandeln, verwenden wir das Distributivgesetz: 
\[
( (A \land B) \lor C = (A \lor C) \land (B \lor C) )
\]
\subsubsection*{Beispiel 2}
\textbf{Ausgangsformel:}
\[
( A \lor (B \land C) )
\]
\textbf{Disjunktive Normalform (DNF):}
Die Formel ist bereits in DNF, da sie eine ODER-Verknüpfung von UND-Verknüpfungen ist.
\textbf{Konjunktive Normalform (KNF):}
Um die Formel in KNF umzuwandeln, verwenden wir das Distributivgesetz: 
\[
( A \lor (B \land C) = (A \lor B) \land (A \lor C) )
\]
\subsubsection*{Beispiel 3}
\textbf{Ausgangsformel:}
\[
( (A \lor B) \land C )
\]
\textbf{Disjunktive Normalform (DNF):}
Um die Formel in DNF umzuwandeln, verwenden wir das Distributivgesetz: 
\[
( (A \lor B) \land C = (A \land C) \lor (B \land C) )
\]
\textbf{Konjunktive Normalform (KNF):}
Die Formel ist bereits in KNF, da sie eine UND-Verknüpfung von ODER-Verknüpfungen ist.
\subsection*{Regeln zur Umstellung}
\subsubsection*{Distributivgesetz:}
\[
( A \land (B \lor C) = (A \land B) \lor (A \land C) )
\]
\[
( A \lor (B \land C) = (A \lor B) \land (A \lor C) )
\]
\subsubsection*{De Morgan'sche Gesetze:}
\[
( \neg (A \land B) = \neg A \lor \neg B )
\]
\[
( \neg (A \lor B) = \neg A \land \neg B )
\]
\subsubsection*{Doppelte Negation:}
\[
( \neg (\neg A) = A )
\]
Diese Regeln helfen dabei, logische Ausdrücke in die gewünschte Normalform zu bringen.
\end{document}
