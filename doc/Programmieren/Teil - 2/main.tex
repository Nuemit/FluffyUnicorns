\documentclass[12pt, a4paper]{article}
\usepackage[a4paper, left=3cm, right=2.5cm, top=2.5cm, bottom=2.5cm]{geometry}
\usepackage{amsmath}
\usepackage{amssymb}
\usepackage{enumitem}
\usepackage[utf8]{inputenc}
\usepackage{ngerman}
\usepackage{listings}
\usepackage{tcolorbox} % Für farbige Boxen
\usepackage[bookmarksopen,colorlinks,plainpages=false,urlcolor=blue,unicode]{hyperref}
\usepackage{xcolor}
\usepackage{setspace}
\usepackage{courier}

\pagenumbering{Roman}
\setcounter{page}{0}

\onehalfspacing % 1.5-facher Zeilenabstand

% % % Farbdefinitionen % % %
\newcommand{\sepa}{\text{\color[HTML]{030303}|}}

% Code-Block-Einstellungen

\lstset{
    language=Python,
    basicstyle=\ttfamily\small,       % Schriftart und -größe
    keywordstyle=\color{blue},       % Schlüsselwörter (z.B. def, while) in Blau
    stringstyle=\color{purple},      
    commentstyle=\color{gray},       % Kommentare in Grau
    numberstyle=\tiny\color{gray},   % Zeilennummern
    numbers=left,                    % Zeilennummern links anzeigen
    stepnumber=1,                    % Jede Zeile nummerieren
    showstringspaces=false,          % Leerzeichen in Strings ignorieren
    frame=single,                    % Rahmen um den Code
    rulecolor=\color{black},         % Rahmenfarbe
    backgroundcolor=\color{lightgray!20}, % Hintergrund leicht grau
    tabsize=4                        % Tab-Breite
}

\lstset{
    language=Java,
    basicstyle=\ttfamily\small\color{white}, % Schriftart, -größe und Farbekeywordstyle=\color{cyan},
    stringstyle=\color[HTML]{ffb10a},
    commentstyle=\color[HTML]{2cc227},
    numberstyle=\tiny\color{gray},
    keywordstyle=\color[HTML]{55abed},
    numbers=left,
    stepnumber=1,
    showstringspaces=false,
    frame=double,
    rulecolor=\color{white},
    backgroundcolor=\color[HTML]{1e211f},
    tabsize=4,
    morekeywords={@Override, public, private, protected, static, final, class, interface, extends, implements, new, return, if, else, for, while, do, switch, case, break, continue, default, try, catch, finally, throw, throws, import, package, this, super, void, int, long, double, float, boolean, char, byte, short, null, true, false}
}

\title{
    \textit{Programmieren 2} \\
        \large Zusammenfassung
}
\author{
    Tim Nätebus \\
    \large \textit{Git: FluffyUnicorns}
}
\date{\today}

\begin{document}
\maketitle
\newpage
{\hypersetup{linkcolor=black}
% or \hypersetup{linkcolor=black}, if the colorlinks=true option of hyperref is used
\tableofcontents
}
\newpage
\section*{Vorwort}
\addcontentsline{toc}{section}{Vorwort}
Dieses Dokument, soll zur Hilfestellung in dem Modul ''Programmieren-1'' dienen.
In Programmieren-1, benutzen wird Java mit der OpenJDK-23.
Dieses Dokument, ist in Kapitel unterteilt, die sich mit den Fortgeschrittenen Themen von Java aus Programmieren-1 beschäftigen. 
Als Wissensbasis ist das Dokument aus \\''\texttt{/doc/Programmieren/Teil-1/main.pdf}'' zu empfehlen.
\section*{Schleifen \& Verzweigungen}
\addcontentsline{toc}{section}{Schleifen \& Verzweigungen}
\subsection*{Schleifen}
\addcontentsline{toc}{subsection}{Schleifen}
% Schleifenkapitel %

\subsubsection*{For-Schleife}
\addcontentsline{toc}{subsubsection}{For-Schleife}
% For-Schleife %
\subsubsection*{For-Each-Schleife}
\addcontentsline{toc}{subsubsection}{For-Each-Schleife}
% For-Each-Schleife %

\subsubsection*{while-Schleife}
\addcontentsline{toc}{subsubsection}{while-Schleife}
% while-Schleife %

\subsubsection*{do-while-Schleife}
\addcontentsline{toc}{subsubsection}{do-while-Schleife}
% do-while-Schleife %


\newpage
\section*{Objekte und Klassen}
\addcontentsline{toc}{section}{Kapitel 1: Objekte und Klassen}
\subsection*{Klassen}
\addcontentsline{toc}{subsection}{Klassen}
% Klassenkapitel %
\subsection*{Objekte}
\addcontentsline{toc}{subsection}{Objekte}
% Objektekapitel %
\subsection*{Spezialität - Access by Reference}
\addcontentsline{toc}{subsection}{Spezialität - Access by Reference}
% Spezialfall Access by Reference %

\end{document}