\documentclass[12pt, a4paper]{article}
\usepackage[a4paper, left=3cm, right=2.5cm, top=2.5cm, bottom=2.5cm]{geometry}
\usepackage{amsmath}
\usepackage{amssymb}
\usepackage{enumitem}
\usepackage[utf8]{inputenc}
\usepackage{ngerman}
\usepackage{listings}
\usepackage{tcolorbox} % Für farbige Boxen
\usepackage{xcolor}
\usepackage{setspace}
\usepackage{courier}

\pagenumbering{Roman}
\setcounter{page}{0}

\onehalfspacing % 1.5-facher Zeilenabstand

% Code-Block-Einstellungen

\lstset{
    language=Python,
    basicstyle=\ttfamily\small,       % Schriftart und -größe
    keywordstyle=\color{blue},       % Schlüsselwörter (z.B. def, while) in Blau
    stringstyle=\color{purple},      
    commentstyle=\color{gray},       % Kommentare in Grau
    numberstyle=\tiny\color{gray},   % Zeilennummern
    numbers=left,                    % Zeilennummern links anzeigen
    stepnumber=1,                    % Jede Zeile nummerieren
    showstringspaces=false,          % Leerzeichen in Strings ignorieren
    frame=single,                    % Rahmen um den Code
    rulecolor=\color{black},         % Rahmenfarbe
    backgroundcolor=\color{lightgray!20}, % Hintergrund leicht grau
    tabsize=4                        % Tab-Breite
}

\lstset{
    language=Java,
    basicstyle=\ttfamily\small\color{white}, % Schriftart, -größe und Farbekeywordstyle=\color{cyan},
    stringstyle=\color[HTML]{ffb10a},
    commentstyle=\color[HTML]{2cc227},
    numberstyle=\tiny\color{gray},
    keywordstyle=\color[HTML]{55abed},
    numbers=left,
    stepnumber=1,
    showstringspaces=false,
    frame=single,
    rulecolor=\color{white},
    backgroundcolor=\color[HTML]{1e211f},
    tabsize=4,
    morekeywords={@Override, public, private, protected, static, final, class, interface, extends, implements, new, return, if, else, for, while, do, switch, case, break, continue, default, try, catch, finally, throw, throws, import, package, this, super, void, int, long, double, float, boolean, char, byte, short, null, true, false}
}

% Titel
\title{
    $Programmieren$\\
    Teil - $1$
    }
\author{
    {Tim Nätebus} \\
    {$Git: FluffyUnicorns$}
}
\date{\today}

\begin{document}
\maketitle
\newpage
\tableofcontents
\newpage
\section*{Variablen}
\addcontentsline{toc}{section}{Variablen}
% % % Variablen Kapitel % % %
\subsection*{Was Genau sind Variablen?}
\addcontentsline{toc}{subsection}{Was Genau sind Variablen?}
Variablen sind Speicherplätze, die einen Wert speichern können. Der Wert kann sich während der Laufzeit des Programms ändern.
% % % Unterkapitel 1 % % %
\subsection*{Welche Datentypen gibt es in Java?}
\addcontentsline{toc}{subsection}{Welche Datentypen gibt es in Java?}
% % % Unterkapitel 2 % % %
Es gibt folgende Datentypen in \textbf{Java}:
\begin{enumerate}[label=-]
    \item \textbf{Primitive Datentypen}
    \item \textbf{Komplexe Datentypen}
\end{enumerate}
\subsubsection*{Primitive Datentypen}
\addcontentsline{toc}{subsubsection}{Primitive Datentypen}
\begin{table}[h!]
    \resizebox{0.8\textwidth}{!}{%
    \begin{tabular}{|l|c|l|}
        \hline
        \textbf{Datentyp} & \textbf{Größe}  & \textbf{Kann darstellen}  \\ \hline
        \textbf{byte}     & \texttt{8 Bit}  & \texttt{Ganze Zahlen}  	\\ \hline
        \textbf{short}    & \texttt{16 Bit} & \texttt{Ganze Zahlen}  	\\ \hline
        \textbf{int}      & \texttt{32 Bit} & \texttt{Ganze Zahlen}  	\\ \hline
        \textbf{long}     & \texttt{64 Bit} & \texttt{Ganze Zahlen}  	\\ \hline
        \textbf{float}    & \texttt{32 Bit} & \texttt{Kommazahlen}      \\ \hline
        \textbf{double}   & \texttt{64 Bit} & \texttt{Kommazahlen}      \\ \hline
        \textbf{char}     & \texttt{16 Bit} & \texttt{Zeichen}          \\ \hline
        \textbf{boolean}  & \texttt{1 Bit}  & \texttt{Wahr \& Falsch}   \\ \hline
        \textbf{void}     & \texttt{-}      & \texttt{Kein Wert}        \\ \hline
    \end{tabular}
    }
\end{table}
\subsubsection*{Komplexe Datentypen}
\addcontentsline{toc}{subsubsection}{Komplexe Datentypen}
\begin{table}[h!]
    \resizebox{0.8\textwidth}{!}{%
    \begin{tabular}{|l|l|}
        \hline
        \textbf{Datentyp} & \textbf{Beschreibung}  \\ \hline
        \textbf{String}   & \texttt{Zeichenkette}   \\ \hline
        \textbf{Array}    & \texttt{Liste von Elementen} \\ \hline
        \textbf{Klasse}   & \texttt{Eigene Datentypen} \\ \hline
    \end{tabular}
    }
\end{table}
\newpage
\subsection*{Was ist ein Literal?}
\addcontentsline{toc}{subsection}{Was ist ein Literal?}
% % % Unterkapitel 3 % % %
Ein Literal, ist einfach ein Wert der einem Datentypen zugeordnet ist. Zum Beispiel:
\begin{enumerate}[label=]
    \item \textbf{25} ist ein Literal vom Datentyp \textbf{int} o. \textbf{long} o. \textbf{short} o. \textbf{byte}
\end{enumerate}
\subsection*{Wie werden welche Variablen deklariert \& initialisiert?}
\addcontentsline{toc}{subsection}{Wie werden welche Variablen deklariert \& initialisiert?}
% % % Unterkapitel 4 % % %
Eine Deklaration der Variable, legt einen Speicherplatz für die Variable fest. Eine Initialisierung, weist der Variable einen Wert zu. \\
Man kann eine Variable innerhalb eines schrittes deklarieren und initialisieren.
\begin{lstlisting}[language=java,title=Beispiele]
    int a;      // Deklaration
    a = 5;      // Initialisierung

    int b = 10; // Deklaration und Initialisierung    
\end{lstlisting}

\end{document}