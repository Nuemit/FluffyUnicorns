\documentclass[12pt, a4paper]{article}
\usepackage[a4paper, left=3cm, right=2.5cm, top=2.5cm, bottom=2.5cm]{geometry}
\usepackage{amsmath}
\usepackage{amssymb}
\usepackage{enumitem}
\usepackage[utf8]{inputenc}
\usepackage{ngerman}
\usepackage{listings}
\usepackage{tcolorbox} % Für farbige Boxen
\usepackage{hyperref}
\usepackage{xcolor}
\usepackage{setspace}
\usepackage{courier}

\pagenumbering{Roman}
\setcounter{page}{0}

\onehalfspacing % 1.5-facher Zeilenabstand

% Code-Block-Einstellungen

\lstset{
    language=Python,
    basicstyle=\ttfamily\small,       % Schriftart und -größe
    keywordstyle=\color{blue},       % Schlüsselwörter (z.B. def, while) in Blau
    stringstyle=\color{purple},      
    commentstyle=\color{gray},       % Kommentare in Grau
    numberstyle=\tiny\color{gray},   % Zeilennummern
    numbers=left,                    % Zeilennummern links anzeigen
    stepnumber=1,                    % Jede Zeile nummerieren
    showstringspaces=false,          % Leerzeichen in Strings ignorieren
    frame=single,                    % Rahmen um den Code
    rulecolor=\color{black},         % Rahmenfarbe
    backgroundcolor=\color{lightgray!20}, % Hintergrund leicht grau
    tabsize=4                        % Tab-Breite
}

\lstset{
    language=Java,
    basicstyle=\ttfamily\small\color{white}, % Schriftart, -größe und Farbekeywordstyle=\color{cyan},
    stringstyle=\color[HTML]{ffb10a},
    commentstyle=\color[HTML]{2cc227},
    numberstyle=\tiny\color{gray},
    keywordstyle=\color[HTML]{55abed},
    numbers=left,
    stepnumber=1,
    showstringspaces=false,
    frame=double,
    rulecolor=\color{white},
    backgroundcolor=\color[HTML]{1e211f},
    tabsize=4,
    morekeywords={@Override, public, private, protected, static, final, class, interface, extends, implements, new, return, if, else, for, while, do, switch, case, break, continue, default, try, catch, finally, throw, throws, import, package, this, super, void, int, long, double, float, boolean, char, byte, short, null, true, false}
}

% Titel
\title{
    $Programmieren$\\
    Teil - $1$
    }
\author{
    {Tim Nätebus} \\
    {$Git: FluffyUnicorns$}
}
\date{\today}

\newpage
\begin{document}
\maketitle
\begin{titlepage}
    \vspace*{\fill}
    \begin{center}
        \Huge
        \textbf{Vorwort}
    \end{center}
    \vspace{1cm}
    \begin{center}
        \large
        Dieses Dokument, soll zur Hilfestellung in dem Modul ''Programmieren-1'' dienen. \\
        \vspace{1cm}
        In Programmieren-1, benutzen wir Java mit der OpenJDK-23. \\
        \vspace{1cm}
        Dieses Dokument, ist in Kapitel unterteilt, die sich mit den Grundlagen von Java beschäftigen. 
    \end{center}
    \vspace*{\fill}
\end{titlepage}
\newpage
\tableofcontents
\newpage
\section*{Variablen}
\addcontentsline{toc}{section}{Variablen}
% % % Variablen Kapitel % % %
\subsection*{Was Genau sind Variablen?}
\addcontentsline{toc}{subsection}{Was Genau sind Variablen?}
Variablen sind Speicherplätze, die einen Wert speichern können. Der Wert kann sich während der Laufzeit des Programms ändern.
% % % Unterkapitel 1 % % %
\subsection*{Welche Datentypen gibt es in Java?}
\addcontentsline{toc}{subsection}{Welche Datentypen gibt es in Java?}
% % % Unterkapitel 2 % % %
Es gibt folgende Datentypen in \textbf{Java}:
\begin{enumerate}[label=-]
    \item \textbf{Primitive Datentypen}
    \item \textbf{Komplexe Datentypen}
\end{enumerate}
\subsubsection*{Primitive Datentypen}
\addcontentsline{toc}{subsubsection}{Primitive Datentypen}
\begin{table}[h!]
    \resizebox{0.8\textwidth}{!}{%
    \begin{tabular}{|l|c|l|}
        \hline
        \textbf{Datentyp} & \textbf{Größe}  & \textbf{Kann darstellen}  \\ \hline
        \textbf{byte}     & \texttt{8 Bit}  & \texttt{Ganze Zahlen}  	\\ \hline
        \textbf{short}    & \texttt{16 Bit} & \texttt{Ganze Zahlen}  	\\ \hline
        \textbf{int}      & \texttt{32 Bit} & \texttt{Ganze Zahlen}  	\\ \hline
        \textbf{long}     & \texttt{64 Bit} & \texttt{Ganze Zahlen}  	\\ \hline
        \textbf{float}    & \texttt{32 Bit} & \texttt{Kommazahlen}      \\ \hline
        \textbf{double}   & \texttt{64 Bit} & \texttt{Kommazahlen}      \\ \hline
        \textbf{char}     & \texttt{16 Bit} & \texttt{Zeichen}          \\ \hline
        \textbf{boolean}  & \texttt{1 Bit}  & \texttt{Wahr \& Falsch}   \\ \hline
        \textbf{void}     & \texttt{-}      & \texttt{Kein Wert}        \\ \hline
    \end{tabular}
    }
\end{table}
\subsubsection*{Komplexe Datentypen}
\addcontentsline{toc}{subsubsection}{Komplexe Datentypen}
\begin{table}[h!]
    \resizebox{0.8\textwidth}{!}{%
    \begin{tabular}{|l|l|}
        \hline
        \textbf{Datentyp} & \textbf{Beschreibung}  \\ \hline
        \textbf{String}   & \texttt{Zeichenkette}   \\ \hline
        \textbf{Array}    & \texttt{Liste von Elementen} \\ \hline
        \textbf{Klasse}   & \texttt{Eigene Datentypen} \\ \hline
    \end{tabular}
    }
\end{table}
\newpage
\subsection*{Was ist ein Literal?}
\addcontentsline{toc}{subsection}{Was ist ein Literal?}
% % % Unterkapitel 3 % % %
Ein Literal, ist einfach ein Wert der einem Datentypen zugeordnet ist. Zum Beispiel:
\begin{enumerate}[label=]
    \item \textbf{25} ist ein Literal vom Datentyp \textbf{int} o. \textbf{long} o. \textbf{short} o. \textbf{byte}
\end{enumerate}
\subsection*{Wie werden Variablen deklariert \& initialisiert?}
\addcontentsline{toc}{subsection}{Wie werden Variablen deklariert \& initialisiert?}
% % % Unterkapitel 4 % % %
Eine Deklaration der Variable, legt einen Speicherplatz für die Variable fest. Eine Initialisierung, weist der Variable einen Wert zu. \\
Man kann eine Variable innerhalb eines schrittes deklarieren und initialisieren.
\begin{lstlisting}[language=java,title=Beispiele]
.
    int a;      // Deklaration
    a = 5;      // Initialisierung

    int b = 10; // Deklaration und Initialisierung    
.
\end{lstlisting}
\subsection*{Konstanten}
Konstanten sind Variablen, die nach Ihrer Initialisierung nicht mehr verändert werden können. \\
Zu Konstanten sind wesentliche dinge zu beachten:
\begin{enumerate}[label=$\alph*)$]
    \item Der Name einer Konstante wird in Großbuchstaben geschrieben.
    \item Der Wert einer Konstante wird mit dem Schlüsselwort \textbf{final} deklariert.
    \item Der Wert einer Konstante wird bei der Deklaration initialisiert.
    \item Der Wert einer Konstante kann nicht mehr verändert werden.
    \item Der Wert einer Konstante kann nur einmal initialisiert werden.
\end{enumerate}
\begin{lstlisting}[language=java,title=Beispiel]
.
    final int MAX = 100; // Konstante Variable
.
\end{lstlisting}
\newpage
\subsection*{Initialisierung von Klassen}
\addcontentsline{toc}{subsection}{Initialisierung von Klassen}
% % % Unterkapitel 5 % % %
Wenn wir Klassen benutzen wollen, müssen wir folgendes Beachten:
\begin{itemize}
    \item Ist die Klasse in der Standardbibliothek enthalten?
    \item Hab Ich das Paket indem die Klasse enthalten ist, installiert?
    \item Muss Ich die Klasse importieren?
\end{itemize}
Aufgrund dessen, das wir nur mit Klassen Arbeiten, die Standardmäßig enthalten sind, wird hier nur auf Standardbibliotheken und Importe eingegangen.
\subsubsection*{Standardbibliotheken}
\addcontentsline{toc}{subsubsection}{Standardbibliothek}
\textbf{Java} bietet eine Vielzahl von Standardbibliotheken an, die wir benutzen können. \\
Einige Beispiele sind:
\begin{enumerate}[label=-]
    \item \textbf{java.util} - Für die Eingabe von Daten
    \item \textbf{java.io} - Für die Ausgabe von Daten
    \item \textbf{java.awt} - Für die Erstellung von GUIs
\end{enumerate}
Die Standardbibliotheken sind zu finden unter \href{https://docs.oracle.com/en/java/javase/23/docs/api/index.html}{Standardbibliotheken-Java-23}
\subsubsection*{Importierung von Klassen}
\addcontentsline{toc}{subsubsection}{Importierung von Klassen}
\begin{lstlisting}[language=java,title=Beispiel mit der Scanner Klasse]
import java.util.Scanner; // Importieren der Scanner Klasse

public class Main {

    public static void main(String[] args) {

        // Initialisierung der Scanner Klasse
        //              mit System.in als Parameter
        Scanner scIn = new Scanner(System.in); 
    
    }
}
\end{lstlisting}
Wir initialisieren die Variable \textbf{scIn} mit der Klasse Scanner als \textbf{komplexen} Datentyp.
\newpage
\section*{Operatoren}
\addcontentsline{toc}{section}{Operatoren}
% % % Operatoren Kapitel % % %
\subsection*{Was sind Operatoren?}
\addcontentsline{toc}{subsection}{Was sind Operatoren?}
Operatoren sind Symbole, die auf Variablen und Werte angewendet werden, um eine Operation durchzuführen.
\subsection*{Welche Arten von Operatoren gibt es?}
\addcontentsline{toc}{subsection}{Welche Arten von Operatoren gibt es?}
% % % Unterkapitel 1 % % %
\textbf{\color[HTML]{a61e27}DISCLAIMER:} \\
\textit{\color[HTML]{a61e27}Die Operatoren, die aufgezählt werden, sind die Operatoren, die wir auch in dem Modul ''Programmieren-1'' benutzen.}
\begin{enumerate}[label=-]
    \item \textbf{Vergleichsoperatoren}
    \item \textbf{Logische Operatoren}
    \item \textbf{Bitweise Operatoren}
    \item \textbf{Zuweisungsoperatoren}
    \item \textbf{Inkrement- \& Dekrementoperatoren}
\end{enumerate}

\end{document}